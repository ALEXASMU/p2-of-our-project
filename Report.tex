\documentclass[12pt]{article}
\usepackage[margin=2cm]{geometry}
\usepackage{amsmath}
\usepackage{graphicx}
\usepackage{color}
\usepackage{listings}



%opening
\title{Projekt – P II}
\title{DM536/DM574: INTRODUCTION TO PROGRAMMING \\ Project – Part II}
\author{By: Alex, Marcus og Tom, Group 3}

\begin{document}

\maketitle

\begin{abstract}
When working with section 2 of the project, we were faced with the task of developing the module known as alquerque.py. This module is what allows the user to play the game alquerque in a terminal. This program starts by asking the users what type of game they want to play, and then it proceeds to initiate. After the program initiates, the game begins, if the players are bots, it tells the user, what moves the bot made. If the moves, however, are by a player, the player must insert their own moves. The way these moves are made are by standard grid format. In this format, rows are represented by letters, and columns are represented by numbers.
\end{abstract}

\section{}

\section{Board_Composer function}
\subsection{Function I/O}
This function takes in a board, of which needs to be displayed, and returns a string that contains a human readable representation of the board.
\subsection{Function internal mechanics}
This function has divided the internal operations into four parts.\\
part 1 handles the conversion of the data acquired through board.py.\\
Part 2 handles the rows which contain the pieces to be drawn.\\
Part 3 handles the rows which cannot contain any pieces.\\
Part 4 prints the characters and numbers that are besides the board.\\

\subsubsection{Part 1}
This part transforms the indekses of the pieces, given by black() and white() from board.py, into a list with 25 elements of which each element reoresents wether or not a piece is at the given position.

\subsubsection{Part 2}
This part prints a character to represent each position, followed by 3 horizontal lines, for each row and column, of note the last columns horizontal lines is specificly not printed.

\subsubsection{Part 3}
This part prints a vertical line under each character, followed by a diagonal line, for each row and column, except the last row.

\subsubsection{Part 3}
This part prints numbers corrosponding to a specific row after each row containing pieces, and after everyother part is done it print the characters corrosponding to each column under every column that contains pieces.\\
This is done to help the user specify which piece they want to move and to where they want it to move.

\subsection{Function testing}
This function has been given the default board afterwhich it has returned this string
"x---x---x---x---x 5\\
| \ | / | \ | / |\\
x---x---x---x---x 4\\
| / | \ | / | \ |\\
x---x--- ---O---O 3\\
| \ | / | \ | / |\\
O---O---O---O---O 2\\
| / | \ | / | \ |\\
O---O---O---O---O 1\\
a   b   c   d   e"\\
which matches how an alquerque board should look\\

\end{document}

